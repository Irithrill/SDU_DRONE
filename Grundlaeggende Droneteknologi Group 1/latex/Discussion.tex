\begin{multicols}{2}
\section{Conclusion}
In conclusion the robot performed sub-par. The design became heaps better with the introduction of the Multi-bot design and was able to mount sensors etc. virtually everywhere. The robot went through some tests at every evolutionary step, which helped defined the shape and functionality of the robot. Gearing of the robot was made too aggressively slow and as a result we had a very robot. the gearing should instead be made opposite so we can increase the speed and adjust the behaviours accordingly. Further tests have been devised for evaluation the new gearing design. 



The planner implementation was found to be performing relatively well, being able to handle complex graphs (167 million nodes was explored for the 2016 competition map). 

It was able to return a greedy solution for most maps very rapidly, because of the optimal datastructures chosen for lookup of identical states and choosing what state to explore next. 

Further work should go into lowering the memory requirements of each state to make it possible to solve even more complex maps. Also, it should be investigated if some parts of the tree can be removed from memory, after it has been explored. 

\end{multicols}


